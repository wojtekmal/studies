\documentclass{../note}

\begin{document}

\begin{exercise}{4.1}
Prove that if
\begin{equation}
P(D_1, \dots, D_m | H_i) = \prod_j P(D_j | H_i), \quad 1 \le i \le n,
\end{equation}
and
\begin{equation}
P(D_1, \dots, D_m | \overline{H_i}) = \prod_j P(D_j | \overline{H_i}), \quad 1 \le i \le n,
\end{equation}
hold with $n > 2$, then at most one of the factors
\begin{equation}
\frac{P(D_1 | H_i)}{P(D_1 | \overline{H_i})}, \dots, \frac{P(D_m | H_i)}{P(D_m | \overline{H_i})}
\end{equation}
is different from unity, therefore at most one of the data sets $D_j$ can produce any updating of the probability for $H_i$.
\end{exercise}

\begin{solution}
Let's consider the case for three hypotheses $H_1, H_2, H_3$ and two events $D_1, D_2$. We use the relation for mutually exclusive $B, C$:

\begin{equation}
P(A | B + C) = \frac{P(B)P(A|B) + P(C)P(A|C)}{P(B) + P(C)}
\end{equation}

$\overline{H_3} = H_1 + H_2$, so (2) becomes

\begin{equation*}
P(D_1D_2|H_1 + H_2) = P(D_1|H_1 + H_2)P(D_2|H_1 + H_2)
\end{equation*}

Using the formula (4), we get

\begin{gather*}
\frac{P(H_1)P(D_1D_2|H_1) + P(H_2)P(D_1D_2|H_2)}{P(H_1) + P(H_2)} =\\
\frac{P(H_1)P(D_1|H_1) + P(H_2)P(D_1|H_2)}{P(H_1) + P(H_2)} \cdot 
\frac{P(H_1)P(D_2|H_1) + P(H_2)P(D_2|H_2)}{P(H_1) + P(H_2)}
\end{gather*}

Or, after using (1), cancelling denominators and rearranging terms:

\begin{gather*}
(P(H_1) + P(H_2))(P(H_1)P(D_1|H_1)P(D_2|H_1) + P(H_2)P(D_1|H_2)P(D_2|H_2)) =\\
(P(H_1)P(D_1|H_1) + P(H_2)P(D_1|H_2))(P(H_1)P(D_2|H_1) + P(H_2)P(D_2|H_2))
\end{gather*}

Expanding and cancelling terms (we can assume that $P(H_i)>0$) gives:

\begin{equation*}
P(D_1|H_1)P(D_2|H_1) + P(D_1|H_2)P(D_2|H_2) = P(D_1|H_1)P(D_2|H_2) + P(D_1|H_2)P(D_2|H_1)
\end{equation*}

This equation is of the form $ab + cd = ac + bd$, which implies $a = c \vee b = d$, so

\begin{equation*}
P(D_1|H_1) = P(D_1|H_2) \vee P(D_2|H_1) = P(D_2|H_2)
\end{equation*}

This is implied by (2) for $i=3$, but it must be true for any $i$. There are three possible values of $i$, so there must be a $j \in \{1, 2\}$ such that the above equation is satisfied for $D_j$ for two different values of $i$. WLOG let it be $j=1$; then we have

\begin{equation*}
P(D_1|H_1) = P(D_1|H_2) = P(D_1|H_3)
\end{equation*}

This in turn implies, using (4):

\begin{gather*}
P(D_1|\overline{H_3}) = P(D_1|H_1 + H_2) = P(D_1|H_3)
\end{gather*}
\end{solution}

\end{document}
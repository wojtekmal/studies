\documentclass{../note}

\begin{document}

\begin{zadanie}{1}
Udowodnij, że dla $n \geq 1$ prawdziwa jest nierówność
\[
2\sqrt{n+1} - 2\sqrt{2} + 1 \leq \sum_{k=1}^{n} \frac{1}{\sqrt{k}} \leq 2\sqrt{n} - 1.
\]
\end{zadanie}
\begin{rozwiazanie}
Dowodzimy to indukcyjnie. Jako bazę wybieramy $n=1$; wtedy te nierówności mówią, że $1 \leq 1 \leq 1$. Krok indukcyjny zaczynamy od następujących nierówności zachodzących dla $n \geq 1$:
\[4n^2 + 4n \leq 4n^2 + 4n + 1\]
\[4n^2 - 4n \leq 4n^2 - 4n + 1\]
Pierwiastkujemy obie strony obu nierówności, a możemy, bo są dodatnie dla $n \geq 1$:
\[2 \sqrt{n(n + 1)} \leq 2n + 1\]
\[2 \sqrt{n(n - 1)} \leq 2n - 1\]
Dzielimy stronami przez $\sqrt{n}$, przerzucamy wyrazy i dostajemy:
\begin{equation}
2\sqrt{n+1} - 2\sqrt{n} \leq \frac{1}{\sqrt{n}} \leq 2\sqrt{n} - 2\sqrt{n-1}
\end{equation}
Z założnenia indukcyjnego dla $n - 1$ wiemy, że:
\begin{equation}
2\sqrt{n} - 2\sqrt{2} + 1 \leq \sum_{k=1}^{n-1} \frac{1}{\sqrt{k}} \leq 2\sqrt{n-1} - 1
\end{equation}

Dodajemy (1) i (2) i dostajemy założenie indukcyjne dla $n$:
\[
2\sqrt{n+1} - 2\sqrt{2} + 1 \leq \sum_{k=1}^{n} \frac{1}{\sqrt{k}} \leq 2\sqrt{n} - 1.
\]
\end{rozwiazanie}

\begin{zadanie}{2}
Udowodnij, że dla $n\ge1$ prawdziwa jest równość
\[\sqrt{2+\sqrt{2+...+\sqrt{2}}}=2\cos\left(\frac{\pi}{2^{n+1}}\right)\]
Z lewej strony równości liczba 2 występuje $n$ razy.
\end{zadanie}
\begin{rozwiazanie}
$0 \leq x \leq \pi$, więc możemy skorzystać ze wzoru $\cos\left(\frac{x}{2}\right) = \sqrt{\frac{1+\cos x}{2}}$. Dowodzimy tezę indukcyjnie. Jako bazę wybieramy $n = 0$; wtedy teza mówi tyle, że $\cos\left(\frac{\pi}{2}\right) = 0$. Krok indukcyjny przeprowadzamy następująco:
\[2\cos\left(\frac{\pi}{2^{n+1}}\right) = 2\sqrt{\frac{1+\cos\left(\frac{\pi}{2^n}\right)}{2}} = \sqrt{2 + 2\cos\left(\frac{\pi}{2^n}\right)} = \sqrt{2+\sqrt{2+...+\sqrt{2}}}\]
W ostatniej równości skorzystaliśmy z założnenia indukcyjnego dla $n - 1$.
\end{rozwiazanie}

\begin{zadanie}{3}
Niech $F_{0}=0$, $F_{1}=1$ oraz $F_{n+1}=F_{n}+F_{n-1}$ dla $n > 1$. Wykazać, że każda dodatnia liczba całkowita może być przedstawiona w postaci sumy parami różnych liczb $F_{n}$.
\end{zadanie}
\begin{rozwiazanie}
Przeprowadzimy indukcję po $n$. Założenie indukcyjne brzmi następująco: wszystkie liczby naturalne mniejsze od $F_{n+1}$ da się przedstawić jako sumę różnych liczb Fibonacciego. Jako bazę bierzemy $n=2$. Wtedy teza mówi, że 1 da się przedstawić w takiej postaci i rzeczywiście $1 = F_0$.\\
Dowodzimy tezę dla $n$. Już wiemy, że liczby mniejsze od $F_n$ da się przedstawić w tej postaci. Pozostają liczby w przedziale $F_n \leq A < F_{n+1}$. Możemy je przedstawić w postaci $A = F_n + B$. $B = A - F_n < F_{n+1} - F_n = F_{n-1} \leq F_n$, więc wiemy, że $B$ jest sumą różnych liczb Fibonacciego mniejszych niż $F_n$, czyli $A$ jest sumą różnych liczb Fibonacciego: tych, z których składa się $B$ oraz $F_n$.
\end{rozwiazanie}
\end{document}
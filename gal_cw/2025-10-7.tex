\documentclass{../note}

\begin{document}
\textbf{Zadanie 2}\\
Dla dowolnej liczby naturalnej $n \ge 2$ definiujemy pierścień reszt z dzielenia przez $n$, jako zbiór
\[
\mathbb{Z}_n = \{0, \ldots, n - 1\}
\]
z operacjami dodawania i mnożenia modulo $n$. Pokaż, że w ten sposób dostajemy pierścień przemienny. Dla jakich wartości $n$ pierścień $\mathbb{Z}_n$ jest ciałem? Odpowiedź uzasadnij.
\textbf{Rozwiązanie}
Pierścień jest przemienny, ponieważ zarówno $a + b (mod n)$, jak i $b + a (mod n)$


\end{document}